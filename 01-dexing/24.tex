
\switchcolumn*[\section{}]

郗公值永嘉丧乱,在乡里,甚穷馁%
\footnote{%
    郗(xī)公:郗鉴,字道徽,以儒雅著名,
                过江后历任兖州刺史、司空、太尉。
    永嘉丧乱:晋怀帝永嘉年间(公元 307--312 年),
              正当八王之乱以后,政治腐败,民不聊生。
              至永嘉五年(公元 311 年),
              在山西称帝的匈奴贵族刘聪(国号汉)将领石勒、刘曜
              俘杀宰相王衍,攻破洛阳,俘怀帝,焚毁全城,史称永嘉之乱。
    穷:生活困难。
    馁(něi):饥饿。
}%
。
乡人以公名德,传共饴之%
\footnote{%
    传:轮流。
    饴(sì):通「饲」,给人吃。
}%
。
公常携兄子迈及外生周翼二小儿往食%
\footnote{%
    迈:字思远,有干世才略。
    周翼:字子卿,历任剡县令、青州刺史、少府卿,享年六十四岁。
    外生:外甥。
}%
。
乡人曰:「
    各自饥困,
    以君之贤,欲共济君耳,
    恐不能兼有所存。
」
公于是独往食,
辄含饭两颊边,还,吐与二儿。
后并得存,同过江%
\footnote{%
    过江:指渡过长江到江南。永嘉之乱时中原人士纷纷过江避难。
          后来镇守建康的琅臣子王司马睿即帝位,
          开始了东晋时代。
}%
。
郗公亡,
翼为剡县,解职归%
\footnote{%
    为剡(shàn)县:指做剡县县令。剡县,古属会稽郡(今浙江嵊县)。
}%
,席苫于公灵床头,心丧终三年%
\footnote{%
    席苫(shān):铺草垫子为席,坐、卧在上面。
                  古时父母死了,就要在草垫子上枕着土块睡,叫做「寝苫枕块。」
    灵床:为死者设置的坐卧用具。
    心丧:好象哀悼父母一样的做法而没有孝子之服。
          古时父母死,服丧三年;
          外亲死,服丧五个月。
          郗鉴是舅父,是外亲,周翼却守孝三年,所以称心丧。
}%
。

%% ----------------------------------------------------------------------------
\switchcolumn

永嘉之乱期间,
郗道徽郗公沦落乡里,穷困潦倒,食不果腹。
乡里人敬重他是个有名有德的贤人,
便让他轮流到家里来吃饭。
郗公经常带着
    他的侄子郗迈与
    外甥周翼
两个孩子一同去吃饭。
乡里人对他说:「
    大家日子都不好过,
    因为敬重您的贤德,
    大伙才挤出粮食来帮忙周济您。
    但是如果还要再带上两个孩子,
    我们恐怕也承受不起了。
」
所以郗公只好独自去吃饭,
吃的时候将饭含在两边的脸颊里,
一等回家就吐出来喂给两个孩子。
后来他们都幸存下来,
一同过了长江。
郗公去世时,
周翼正任剡县县令,
听说之后辞官回来,
在郗公灵床前寝苫枕块,
像哀悼自己的父母一般
为郗公守了足足三年的丧。

