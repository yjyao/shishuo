
\switchcolumn*[\section{}]

王恭从会稽还,
王大看之%
\footnote{%
    王恭:字孝伯,为官清廉,
          晋安帝时起兵反对帝室,
          被杀。
    会稽:郡名,郡治在今浙江省绍兴县。
    王大:王忱,小名佛大,也称阿大,
          是王恭的同族叔父辈。
          官至荆州刺史。
}%
。
见其坐六尺簟%
\footnote{%
    蕈(diàn):竹席。
}%
,
因语恭:「
    卿东来,
    故应有此物,
    可以一领及我。
」
恭无言。
大去后,
即举所坐者
送之。
既无余席,便坐荐上%
\footnote{%
    荐:草席。
}%
。
后大闻之,
甚惊,曰:「
    吾本谓卿多,故求耳。
」
对曰:「
    丈人不悉恭,
    恭作人无长物%
    \footnote{%
        长(zhàng)物:多余的东西。
    }%
    。
」

%% ----------------------------------------------------------------------------
\switchcolumn

王恭从会稽回来,
他的叔叔王大去看望他。
王大看到他坐在一张六尺见方的竹席上,
非常喜欢,
就对王恭说:「
    你从东边回来,
    难怪有这样的好东西。
    要不你拿一领给我吧?
」
王恭没说什么。
等到王大走后,
王恭就把自己坐的那张席子
送给了王大。
但这么一来,
自己就没有席子坐了,
于是便找了一张草席坐。
后来王大听说了这件事,
非常震惊。
他告诉王恭说
他本以为王恭有很多那样的席子
才和他要的。
王恭回答他说:「
    叔叔你不了解我,
    我这个人
    从来不留多余的东西。
」

