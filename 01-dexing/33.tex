
\switchcolumn*[\section{}]

谢奕作剡令%
\footnote{%
    谢奕:字无奕,东晋名臣谢安的哥哥,年少时便气宇不凡。
    令:指县令,一县的行政长官。
}%
,
有一老翁犯法,
谢以醇酒罚之,乃至过醉,而尤未已。
太傅时年七八岁,着青布绔,在兄膝边坐%
\footnote{%
    太傅:指谢安,字安石,为人温雅畅达,学行俱佳。
    绔:即裤。
}%
,谏曰:「
    阿兄,老翁可念,何可作此%
    \footnote{%
        念:怜悯;同情。
    }%
    !
」
奕于是改容曰:「
    阿奴欲放去邪%
    \footnote{%
        阿奴:长对幼、尊对卑的称呼。
    }%
    ?
」
遂遣之。

%% ----------------------------------------------------------------------------
\switchcolumn

谢奕任剡县县令的时候,
有一位老人犯了法,
谢奕就罚他喝烈酒,
老头已经醉得不成样子了,
谢奕还不肯罢休。
谢安石谢太傅当时才七八岁,
穿着青布套裤,
正坐在哥哥谢奕的膝上。
看到老人的惨状,
谢安石劝住哥哥说:「
    哥哥,
    你看这老人家那么可怜,
    何苦这样为难他呢!
」
谢奕的脸色缓和下来,说:「
    那你是想要放他走咯?
」
于是就打发老人离开了。

