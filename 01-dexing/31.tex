
\switchcolumn*[\section{}]

庾公乘马有的卢%
\footnote{%
    庾公:庚亮,字元规,任征西大将军、荆州刺史。
    的卢:又作的颅,是额上有白色斑点的马,古人认为这是凶马,它的主人会得祸。
}%
,
或语令卖去,
庾云:「
    卖之必有买者,即当害其主,宁可不安己而移于他人哉?
    昔
    孙叔敖杀两头蛇以为后人,古之美谈%
    \footnote{%
        孙叔敖:春秋时代楚国的令尹。
                据贾谊《新书》载,
                孙叔敖小时候在路上看见一条两头蛇,回家哭着对母亲说:
                「听说看见两头蛇的人一定会死,我今天竟看见了。」
                母亲问他蛇在哪里,
                孙叔敖说:
                「我怕后面的人再见到它,就把它打死埋掉了。」
                他母亲说:
                「你心肠好,一定会好心得好报,不用担心。」
    }%
    。
    效之,不亦达乎?
」

%% ----------------------------------------------------------------------------
\switchcolumn

庾公庾亮有一匹的卢马,
传说的卢是凶马,会带来不详,
有人就劝庾公将它卖出。
庾公说:「
    卖马的话必然就会有买主,
    那他就要遭殃。
    哪有这样损人利己的道理呢?
    从前
    孙叔敖为了后来人着想
    杀死了双头蛇,
    这个故事
    历来为人称道。
    今天我效仿孙叔敖,
    也算潇洒!
」

