
\switchcolumn*[\section{}]

祖光禄少孤贫,
性至孝,常自为母炊爨作食%
\footnote{%
    祖光禄:祖纳,字士言,东晋时任光禄大夫。
    炊爨(cuàn):生火做饭。
}%
。
王平北闻其佳名,以两婢饷之,因取为中郎%
\footnote{%
    王平北:王乂(yì),字叔元,曾任平北将军。
    饷:赠送。
    婢:受役使的女子。
    取:任用。
    中郎:近侍之官,担任护卫、侍从,所以下文戏称为奴。
}%
。
有人戏之者曰:「
    奴价倍婢。
」
祖云:「
    百里奚亦何必轻于五之皮邪%
    \footnote{%
        百里奚(xī):人名,姜姓,百里氏,名奚。
                      关于百里奚,历史上有不同记载。
                      据《史记·秦本纪》载,
                      百里奚是春秋时虞国大夫,
                      晋国灭虞国时俘虏了他。
                      逃跑后,被楚国人抓住,
                      秦穆公听说他有才德,就用五张羊皮赎了他,授以国政,
                      号为五羖大夫。
                      羖(gǔ):黑色的公羊。
    }%
    ?
」

%% ----------------------------------------------------------------------------
\switchcolumn

光禄大夫祖纳幼年丧父,家境贫寒,
但他非常孝顺,
常常自己为母亲下厨做饭。
平北将军王乂听说以后,
送他婢女两名,
任他为中郎。
有人戏谑祖纳说:「
    我听说一个男仆只值两个婢女?
」
祖纳反驳道:「
    照你这么说,
    百里奚也只值五张羊皮咯?
」

