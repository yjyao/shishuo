
\switchcolumn*[\section{}]

顾荣在洛阳,
尝应人请%
\footnote{%
    顾荣:字彦先,东吴丞相顾雍之孙,
          西晋末年拥护司马氏政权南渡的江南士族首脑。
          弱冠之后就仕于孙吴,吴亡后,与陆机、陆云同入洛,号为“洛阳三俊”。
          后为琅玡王司马睿(晋元帝)安东军司,加散骑常侍,
          司马睿但凡有谋划,都与顾荣商议。
}%
,
觉行炙人有欲炙之色,因辍己施焉%
\footnote{%
    行炙人:传递菜肴的仆役。
    炙:烤肉。
}%
,
同坐嗤之%
\footnote{%
    嗤(chī):讥笑。
}%
。
荣曰:「
    岂有终日执之,而不知其味者乎?
」
后遭乱渡江,每经危急,常有一人左右己%
\footnote{%
    左右:帮助。
}%
,
问其所以,乃受炙人也。

%% ----------------------------------------------------------------------------
\switchcolumn

% %% Jy

% %% 妖

