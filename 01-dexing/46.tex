
\switchcolumn*[\section{}]

孔仆射为孝武侍中,
豫蒙眷接%
\footnote{%
    孔仆射:孔安国,字安国。
            自小家境贫苦。
            他以儒学素养著称。
            晋孝武帝时很受待见,官至尚书仆射。
    豫:喜悦;幸福。
    眷接:恩宠和接待。
}%
。
烈宗山陵%
\footnote{%
    烈宗:晋孝武帝庙号,即死后立室奉祀时起的名号。
    山陵:帝王的坟墓;这里指归山陵,即死。
}%
,
孔时为太常,
形素羸瘦,
着重服,
竟日
涕泗流涟,
见者以为真孝子%
\footnote{%
    重服:孝服中之重者,即父母丧时所穿的孝服。
}%
。

%% ----------------------------------------------------------------------------
\switchcolumn

% %% Jy
% 尚书仆射孔安国
% 在孝武帝时
% 任侍中的时候,
% 有幸受到孝武帝的厚待。
% 孝武帝驾崩时,
% 孔安国正在做太常,
% 他本来身形就消瘦,
% 这时候更穿着重孝服,
% 整日地痛哭流涕,
% 别人看到他这副模样
% 都以为是他自己的生父生母过世了。

% %% 妖
% 孔安国担任侍中的时候,
% 非常受晋孝武帝的赏识。
% 孝武帝驾崩后,
% 此时官居太常的孔安国
% 形容消瘦,
% 穿着重孝孝服,
% 每天都泪流满面,
% 见到他的人都认为他是真孝子。
