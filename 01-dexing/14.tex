
\switchcolumn*[\section{}]

王祥事后母朱夫人甚谨%
\footnote{%
    王祥:字休征,魏晋时人,是个孝子。
          因为侍奉后母,年纪很大才进入仕途,官至太常、太保。
}%
。
家有一李树,结子殊好,母恒使守之。
时风雨忽至,祥抱树而泣。
祥尝在别床眠,母自往暗斫之%
\footnote{%
    斫(zhuó):大锄;引申为用刀、斧等砍。
}%
。
值祥私起%
\footnote{%
    私:小便。
}%
,空斫得被。
既还,知母憾之不已,因跪前请死。
母于是感悟,爱之如己子。

%% ----------------------------------------------------------------------------
\switchcolumn

王祥总是尽心尽力地侍奉他的继母朱夫人。
他们家有一棵李树,结的果子都很好,
朱夫人就吩咐王祥去照看它。
有时遇到刮风下雨,
王祥就抱住李树痛哭流涕。

有一次,王祥在别床睡,
继母去暗杀他,
正巧王祥起夜,
只砍到了一床空被。
王祥回来之后,
明白事情经过,
心知继母心里憾恨不已,
于是就跪在继母身前
请求一死。
继母深受感动,
幡然醒悟,
从这以后
对王祥视如己出。

