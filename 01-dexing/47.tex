
\switchcolumn*[\section{}]

吴道助、附子兄弟居在丹阳郡后%
\footnote{%
    吴道助、附子:吴坦之,字处靖,小名道助;
                  吴隐之,字处默,小名附子。
                  据《晋安帝纪》载,
                  吴隐之既孝顺,又清廉,
                  奉䘵薪水都与族人同享,
                  冬天也盖不上厚被子。
                  传言广州有一贪泉,
                  喝下泉水的人就会贪得无厌。
                  吴隐之任广州刺史时,
                  得意去贪泉喝了一瓢,
                  赋诗说自己哪怕喝了贪泉
                  也不会变心易性。
}%
,
遭母童夫人艰,
朝夕哭临%
\footnote{%
    哭临(lìn):哭吊死者的哀悼仪式。
}%
。
及思至%
\footnote{%
    思至:疑为「周忌」之误。
}%
,
宾客吊省,
号踊哀绝,路人为之落泪。
韩康伯时为丹阳尹,
母殷在郡,
每闻二吴之哭,
辄为凄恻,
语康伯曰:「
    汝若为选官%
    \footnote{%
        选官:人力资源主管。
    }%
    ,
    当好料理此人。
」
康伯亦甚相知%
\footnote{%
    知:要好。
}%
。
韩后果为吏部尚书%
\footnote{%
    吏部尚书:吏部的行政长官。
              吏部掌管官吏的任免、考核、升降等。
}%
。
大吴不免哀制%
\footnote{%
    不免哀制:指经不起丧亲的悲痛而死。
}%
,
小吴遂大贵达。

%% ----------------------------------------------------------------------------
\switchcolumn

吴坦之和吴隐之兄弟俩
居住在丹阳郡府后面。
他们的母亲童夫人过世,
兄弟俩没日没夜地哭悼。
宾客前来吊唁时,
兄弟二人
号啕大哭,捶胸顿足,恸痛欲绝。
连路人都不禁落泪。

当时
韩康伯在丹阳郡任府尹,
他的母亲殷氏在丹阳郡府中,
每次听见两兄弟的哭声,
都觉得凄惨难耐,心生可怜。
她嘱咐韩康伯,
如果做了选拔人才的官员,
一定要好好照顾这两兄弟。
韩康伯与他们关系也非常交好。
后来
韩康伯果真出任了吏部尚书,
掌管官吏的任免升降。
那个时候,
吴坦之已经因为哀痛过度,不幸离世了;
好在吴隐之还在人间,
得以受到重用,飞黄腾达。
