
\switchcolumn*[\section{}]

吴道助、附子兄弟居在丹阳郡%
\footnote{%
    吴道助、附子:吴坦之,字处靖,小名道助;
                  吴隐之,字处默,小名附子。
                  据《晋安帝纪》载,
                  吴隐之既孝顺,又清廉,
                  奉䘵薪水都与族人同享,
                  冬天也盖不上厚被子。
                  传言广州有一贪泉,
                  喝下泉水的人就会贪得无厌。
                  吴隐之任广州刺史时,
                  得意去贪泉喝了一瓢,
                  赋诗说自己哪怕喝了贪泉
                  也不会变心易性。
}%
,
后遭母童夫人艰,
朝夕哭临%
\footnote{%
    哭临(lìn):哭吊死者的哀悼仪式。
}%
。
及思至%
\footnote{%
    思至:疑为「周忌」之误。
}%
,
宾客吊省,
号踊哀绝,路人为之落泪。
韩康伯时为丹阳尹,
母殷在郡,
每闻二吴之哭,
辄为凄恻,
语康伯曰:「
    汝若为选官%
    \footnote{%
        选官:人力资源主管。
    }%
    ,
    当好料理此人。
」
康伯亦甚相知%
\footnote{%
    知:要好。
}%
。
韩后果为吏部尚书%
\footnote{%
    吏部尚书:吏部的行政长官。
              吏部掌管官吏的任免、考核、升降等。
}%
。
大吴不免哀制%
\footnote{%
    不免哀制:指经不起丧亲的悲痛而死。
}%
,
小吴遂大贵达。

%% ----------------------------------------------------------------------------
\switchcolumn

% %% Jy
% 吴坦之和吴隐之兄弟俩
% 居住在丹阳郡。
% 他们的母亲童夫人过世,
% 兄弟俩没日没夜地哭悼。
% 一年后,
% 到了童夫人的忌日那天,
% 四方宾客都来哀吊童夫人。
% 兄弟二人
% 号啕大哭,捶胸顿足,恸痛欲绝。
% 连路人都不禁落泪。
%
% 韩康伯当时是丹阳郡的府尹,
% 他的母亲殷氏也在丹阳,
% 每次听闻两兄弟的哭声,
% 都觉得凄惨难耐,心生可怜。
% 她还让韩康伯有机会的时候
% 一定要好好提拔重用这两兄弟。
% 韩康伯本身也与他们非常交好。
% 后来
% 韩康伯果然做了吏部尚书,
% 掌管官吏的任免升降。
% 那个时候,
% 吴坦之已经不禁哀痛,离开人世;
% 好在吴隐之还在人间,
% 得以受到重用,乘黄腾达。

% %% 妖
% 吴坦之,吴隐之兄弟俩住在丹阳郡官署后面,
% 他们母亲去世了,
% 就日夜痛哭。
% 思念备至,
% 有宾客来悼唁时,
% 他们就顿足大哭,嚎啕不已,可谓闻者落泪见者伤心。
% 这时时韩康伯任丹阳尹,
% 他的母亲殷氏住在郡府中,
% 每次听到吴家兄弟俩的哭声,
% 总是辗转反侧,
% 就对康伯说:“
%  你如果做了选官,
%  应该好好照顾这两个人。”
% 韩康伯也和他们关系很好。
% 后来韩康伯果然出任吏部尚书。
% 这时吴坦之因为哀毁过度,不幸离世,
% 弟弟隐之却因康伯而富贵显达。
