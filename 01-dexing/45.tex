
\switchcolumn*[\section{}]

吴郡陈遗,
家至孝。
母好食铛底焦饭,
遗作郡主簿,
恒装一囊,
每煮食,
辄伫录焦饭,
归以遗母。
后
值孙恩贼出吴郡,
袁府郡即日便征。
遗已聚敛得数斗焦饭,
未展归家,
遂带以从军。
战于沪渎,
败。
军人溃散,
逃走山泽,
皆多饥死,
遗独以焦饭得活。
时人以为纯孝之报也。

%% ----------------------------------------------------------------------------
\switchcolumn

% %% Jy

% %% 妖
% 吴郡有一个叫陈遗的人,
% 在家里非常孝顺。
% 他母亲喜欢吃锅巴,
% 陈遗做吴郡主簙时,
% 总是在身上揣着一个口袋,
% 他每每做饭后,
% 就把锅巴放进这个口袋里,
% 等回家之后给他母亲。
% 后来,
% 遇见孙恩的贼兵入侵吴郡,
% 袁山松马上就要率军征讨。
% 这是陈遗已经攒了几斗锅巴,
% 来没来得及回家,
% 就只得带着它们出征。
% 两军相战于沪渎,
% 袁军大败。
% 士卒四散奔逃,
% 进了山林沼泽,
% 大部分人饿死了,
% 只有陈遗靠着锅巴活了下来。
% 当时的人说这是他的孝行所应得的。
