
\switchcolumn*[\section{}]

晋简文为抚军时%
\footnote{%
    晋简文:晋简文帝司马昱(yù),字道万,
            晋元帝司马睿的幼子,
            即位前封会稽王,任抚军将军,
            后又进位抚军大将军、丞相。
}%
,
所坐床上,尘不听拂%
\footnote{%
    床:坐具。古时候卧具坐具都叫床。
    听:听凭;任凭。
}%
,
见鼠行迹,视以为佳。
有参军见鼠白日行,以手板批杀之,
抚军意色不说%
\footnote{%
    参军:官名,是将军幕府所设的官。
    手板:即「笏(hù)」,
          下属谒见上司时所拿的狭长板子,上面可以记事。
          魏晋以来习惯执手板。
    说:通「悦」。
}%
。
门下起弹%
\footnote{%
    门下:门客,贵族家里养的帮闲人物。
    弹:弹劾。
}%
。
教曰:「
    鼠被害,
    尚不能忘怀,
    今复以鼠损人,无乃不可乎%
    \footnote{%
        无乃:恐怕。
    }%
    ?
」

%% ----------------------------------------------------------------------------
\switchcolumn

% %% Jy

% %% 妖

