
\switchcolumn*[\section{}]

殷仲堪既为荆州%
\footnote{%
    殷仲堪:表字不详,
            笃信天师道,生活俭省,可是事神不借钱财。
}%
,
值水俭%
\footnote{%
    水俭:因水灾而年成不好。
    俭:歉收。
}%
,
食常五碗,外无余肴,
饭粒脱落盘席间,辄拾以啖之%
\footnote{%
    啖(dàn):吃。
}%
。
虽欲率物,亦缘其性真素
。
每语子弟云:「
    勿以我受任方州,
    云我豁平昔时意%
    \footnote{%
        豁(huò):抛弃。
        时意:时俗。
    }%
    ,
    今吾处之不易。
    贫者,士之常,
    焉得登枝而捐其本?
    尔曹其存之%
    \footnote{%
        曹:等,辈;尔曹,你们。
        其:表命令、劝告的语气副词。
    }%
    !
」

%% ----------------------------------------------------------------------------
\switchcolumn

殷仲堪任荆州刺史的时候,
正值涝灾,
每天饭不多吃,
也从不加菜。
有饭粒掉到盘子里、座席上,
都要捡起来吃掉。
他这么做,
一方面是想树立表率,
另一方面也是因为他生性就非常俭朴。
他常常教导晚辈们说:「
    别因为我现在做了刺史,
    就以为我要扔下了平常的作风,
    到现在我也没有丝毫改变。
    读书人,就是要安贫乐道,
    不能说你登上树梢变了凤凰了,
    就忘了本了。
    你们都要牢牢地记住这一点。
」
