
\switchcolumn*[\section{}]

王安丰遭艰,至性过人%
\footnote{%
    王安丰:指王戎。
    艰:父母丧。
    至性:纯真的天性。
}%
。
裴令往吊之,曰:「
    若使一恸果能伤人,
    濬冲必不免灭性之讥%
    \footnote{%
        灭性:指过度哀伤而危及性命。
              《孝经·丧亲章》「毁不灭性,圣人之教也。」
              所以古人认为哀伤过度而丧命是不符合圣人之教的。
              《曲礼》「
                  居丧之礼,毁瘠不形,视听不衰,不胜丧,乃比于不慈不孝。
              」
    }%
    。
」

%% ----------------------------------------------------------------------------
\switchcolumn

王戎丧母,
哀恸不已,
身边的人都不禁替他担心。
中书令裴楷前去吊唁,对他说:「
    都说『毁不灭性,圣人之教也』,
    假如哀痛过度果真能损害到人的健康,
    那么濬冲你免不了要受到世人的非议,
    说你灭性了。
」
