
\switchcolumn*[\section{}]

周镇罢临川郡还都,
未及上住,泊青溪渚,
王丞相往看之%
\footnote{%
    周镇:字康时,东晋时陈留尉氏人,清正廉明,颇有政声。
    王丞相:王导,字茂弘。
            曾辅助晋元帝经营江左,后任丞相。
}%
。
时夏月,暴雨卒至,舫至狭小,而又大漏,殆
无复坐处%
\footnote{%
    卒:通猝,突然。
    舫(fǎng):船。
}%
。
王曰:「
    胡威之清,何以过此%
    \footnote{%
        胡威:字伯虎。
              胡威的父亲胡质为官清廉,
              做荆州刺史时,胡威从京都去看他。
              胡威回家时,他只给了一匹绢做口粮钱。
              胡威一路上自己打柴做饭。
              后来胡质手下一个都督在途中装做路人,资助胡威。
              胡威问明情况,得知是都督以后,把那匹绢给了他,让他走了。
              胡威到家后把这事告诉了父亲。
              胡质认为这有损于自己的清廉,
              就把那个都督抓来打了一百棍,把他开除了。
              又有一次,胡威做徐州刺史时,
              受到晋世祖接见。
              世祖感叹胡质清廉之余,
              问胡威与父亲谁更清廉,
              胡威说自己比不上父亲。
              世祖问其原因,
              胡威说:「
                  下臣的父亲生怕别人知道自己有多清廉;
                  下臣我则生怕别人不知道自己有多清廉。
              」
    }%
    !
」
即启用为吴兴郡。

%% ----------------------------------------------------------------------------
\switchcolumn

% %% Jy
% 周镇从临川解官回到京都的时候,
% 还没上岸,
% 先把船泊在了青溪渚。
% 王丞相便去看望他。
% 当时正值夏季,
% 突然间下起暴雨,
% 周镇乘的船本身就小得很,
% 又处处漏雨,
% 简直没有个落脚的地方。
% 王导感叹:「
%     人说胡威清廉,我看也不过如此吧!
% 」
% 于是任用周镇做了吴兴的太守。

% %% 妖

