
\switchcolumn*[\section{}]

邓攸始避难,于道中弃己子,全弟子%
\footnote{%
    邓攸:字伯道,行政清明,深得民心,被誉为「中兴良守」。
          他弟弟早死,留下一个儿子由邓攸抚养,
          邓攸渡江后,终生也没能再生一个儿子,
          后人常叹「天道无知,使邓伯道无儿」。
}%
。
既过江,取一妾,甚宠爱。
历年后,讯其所由,
妾具说是北人遭乱,
忆父母姓名,乃攸之甥也。
攸素有德业,言行无玷,闻之哀恨终身,遂不复畜妾%
\footnote{%
    玷(diàn):污点;过失。
}%
。

%% ----------------------------------------------------------------------------
\switchcolumn

% %% Jy
% 邓攸避永嘉之乱时,
% 在逃亡途中
% 抛弃了自己的儿子
% 而保全了弟弟留下来的遗孤。
% 过江以后,
% 取了一个小妾,
% 对她宠爱有加。
% 一年之后,
% 有一次问起小妾的身世,
% 小妾详尽地讲述了自己的来由。
% 原来她也是来自江北、
% 永嘉之乱时逃难到江南来的。
% 听说了她父母的姓名之后,
% 邓攸才发现原来她可算是自己的远房亲戚。
% 邓攸平日里功德著赫,
% 言谈行止都清白无瑕;
% 得知自己无意中竟犯下乱伦之过,
% 悔痛终身。
% 自此以后便再不娶妾。

% %% 妖
% 许攸躲避永嘉之难时,在路上通过丢弃自己的儿子来使侄子存活。
% 待到了江南,他纳了一房妾室,宠爱有加。
% 一年之后,许攸询问她的遭遇。
% 小妾详细回忆了她的经历,说自己是北方人遭难流落至此,回忆起父母的姓名,发现原来是许攸的外甥女。
% 许攸素来品德高洁,言谈举止都没有瑕疵,听到这事悔恨终身,于是再不纳妾。
