
\switchcolumn*[\section{}]

刘尹在郡,临终绵惙%
\footnote{%
    刘尹:刘惔(tán),字真长,任丹阳尹,即京都所在地丹阳郡的行政长官。
    绵惙(chuǒ):气息微弱,指奄奄一息。
}%
,
闻阁下祠神鼓舞%
\footnote{%
    阁:供神佛的地方。
    祠:祭祀,为除病祷告。
    鼓舞:击鼓舞蹈,祭神的一种仪式。
}%
,
正色曰:「
    莫得淫祀%
    \footnote{%
        淫祀:滥行祭祀。不该祭祀而祭祀,即不合礼制的祭祀,叫淫祀。
    }%
!」
外请杀车中牛祭神%
\footnote{%
    车中牛:驾车的牛。
            晋代常坐牛车,杀驾车的牛来祭祀是常事。
}%
,
真长曰:「
    丘之祷久矣,勿复为烦%
    \footnote{%
        丘之祷久矣:这句话出自《论语·述而》。
                    一次,孔子(名丘)得了重病,
                    他的弟子子路请求允许向神祷告,
                    孔子说:「丘之祷久矣(我早就祷告过了)」,
                    委婉拒绝了子路的请求。
                    刘惔喜欢老庄之学,纯任自然,所以不想祭神。
    }%
!」

%% ----------------------------------------------------------------------------
\switchcolumn

% %% Jy

% %% 妖

