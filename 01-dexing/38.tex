
\switchcolumn*[\section{}]

范宣年八岁,
后园挑菜,误伤指,大啼%
\footnote{%
    范宜:字子宣,家境贫寒,崇尚儒家经典,以讲授儒学为业。
          居住在豫章郡,被召为太学博士、散骑郎,推辞不就。
    挑:挑挖;挖出来。
}%
。
人问:「痛邪?」
答曰:「
    非为痛,
    身体发肤,不敢毁伤%
    \footnote{%
        身体发肤,不敢毁伤:
            语出《孝经》「身体发肤,受之父母,不敢毁伤,孝之始也。」
            身,躯干;体,头和四肢。
    }%
    ,
    是以啼耳。
」
宣洁行廉约,
韩豫章遗绢百匹,不受%
\footnote{%
    韩豫章:韩伯,字康伯,历任豫章太守、丹杨尹、吏部尚书。
    遗(wèi):赠送。
}%
;
减五十匹,复不受。
如是减半,遂至一匹,既终不受。
韩后与范同载,
就车中裂二丈与范,云:「
    人宁可使妇无裈邪%
    \footnote{%
        裈(kūn):裤子。
    }%
?」
范笑而受之。

%% ----------------------------------------------------------------------------
\switchcolumn

范宣八岁的时候,
在后院挖菜,
一不留神伤到了手指,
就大哭起来。
旁人就问他:「
    是不是很痛啊?
」
范宣回答说:「
    我并不是因为疼才哭的,
    只是身体发肤,受之父母,不敢毁伤。
    我哭是因为弄伤了手指
    是大不孝啊。
」

范宣为人廉洁勤俭,
豫章太守韩康伯
曾赠与他百匹绢缎,
范宣却不愿意接受。
韩太守减下了五十匹,
他还是不要。
就这样反复减半,
直到只剩下一匹,
范宣依然不为所动。
后来
他们二人同乘一车,
韩太守在车中
扯下二丈布来,
递给范宣,
说:「
    不管怎么说,
    您总不能让尊夫人
    连条裤子都没得穿吧?
」
范宣笑了,
这才收下了布。

