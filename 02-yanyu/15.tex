
\switchcolumn[0]*[\section{}]

嵇中散语赵景真%
\footnote{%
    嵇中散:指嵇康。
    赵景真:
        赵至,字景真,
        有口才,
        曾任辽东郡从事,主持司法工作,
        以清当见称。
}%
:「
    卿
    瞳子白黑分明,
    有白起之风,
    恨量小狭%
    \footnote{%
        白起:
            战国时秦国的名将,
            封武安君。
            据说他瞳子白黑分明。
            人们认为,
            这样的人一定见解高明。
    }%
    。
」
赵云:「
    尺表能审玑衡之度,
    寸管能测往复之气%
    \footnote{%
        表:用来观测天象的一种标竿。
        玑衡:古代测量天象的仪器,即浑天仪。
        管:指古代用来校正乐律的竹管。
    }%
    。
    何必在大,
    但问识如何耳。
」

%% ----------------------------------------------------------------------------
\switchcolumn

% %% Jy
% 嵇康对赵景真说:「
%     我看你
%     眼睛黑白分明,
%     颇有传说中白起的神韵嘛。
%     只可惜眼睛整体小了点。
% 」
% 赵景真说:「
%     几尺长的表尺就能测定浑天仪的度数,
%     几寸长的管子就能测量乐器的音准。
%     大小有什么要紧的呢?
%     关键哪还是要看见识的长短。
% 」

% %% 妖
% 嵇康对赵景真说:“
%  你的眼睛黑白分明,
%  有名将白起的风采,
%  只可惜小了点。”
% 赵景真回答说:“
%  不长的尺子能够衡量天象的变化,
%  短小的竹管也能测出四时的循环。
%  何必在乎大小,
%  只要询问学识如何就可以了。”
