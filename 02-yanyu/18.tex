
\switchcolumn[0]*[\section{}]

嵇中散既被诛,
向子期举郡计入洛%
\footnote{%
    向子期:
        向秀,
        字子期,
        和嵇康很友好,
        标榜清高。
        嵇康被杀后,
        他便改变初衷,出来做官。
        到京城后,去拜访大将军司马昭。
        这里记的就是他和司马昭的一段对话。
    郡计:
        计是计薄、帐簿,
        列上郡内众事的。
        按:
        汉制,每年年末,
        太守派遣椽、吏各一人为上计簿使,
        呈送计簿到京都汇报。
}%
,
文王引进,
问曰:「
    闻君有箕山之志,
    何以在此%
    \footnote{%
        箕山:
            山名,在今河南省登封县东南。
            尧时巢父、许由在箕山隐居。
            这里说箕山之志,就是指归隐之志。
    }%
    ?
」
对曰:「
    巢、许狷介之士,
    不足多慕%
    \footnote{%
        狷(juàn)介:
            孤高;洁身自好。
        多慕:
            称赞、羡慕。
    }%
    。
」
王大咨嗟%
\footnote{%
    咨嗟:赞叹。
}%
。

%% ----------------------------------------------------------------------------
\switchcolumn

% %% Jy

% %% 妖
% 中散大夫嵇康被杀后,
% 他的朋友向子期送郡里的账簿到洛阳去汇报。
% 文王推荐了他,
% 问道:“
%  听说您有隐逸的志向,
%  为什么会来这里呢?”
% 向子期回答说:“
%  巢父、许攸这样孤高的人,
%  并不值得称赞。”
% 文王听了,大为赞赏。
