
\switchcolumn[0]*[\section{}]

张玄之、顾敷是顾和中外孙%
\footnote{%
    张玄之:
        字祖希,
        少年有学识,
        任冠军将军、吴兴太守。
        名亚谢玄,
        时人把他们并称为北南二玄。
    中外孙:孙子和外孙。
}%
,
皆少而聪惠,
和并知之,
而常谓顾胜。
亲重偏至,
张颇不厌%
\footnote{%
    不厌:不满意。
}%
。
于时
张年九岁,
顾年七岁,
和与俱至寺中,
见佛般泥洹像%
\footnote{%
    般泥洹(bō ni huán)像:
        卧佛像。
        般泥洹,即洹槃,
        佛教用语,
        指修行的最高境界,
        也称僧尼死亡。
}%
,
弟子
  有泣者,
有不泣者。
和以问二孙。
玄谓%
\footnote{%
    玄:指玄之。
}%
:「
      被亲故泣,
    不被亲故不泣%
    \footnote{%
        多本作「
            彼亲故泣,
            彼不亲故不泣%
        」。
    }%
    。
」
敷曰:「
    不然。
    当由
        忘情故不泣,
    不能忘情故泣%
    \footnote{%
        忘情:
            指哀乐不动于心;
            佛教中称已经习得佛法,并能断惑的人,
            也就是无可再学之人,
            为「无学者」。
            《大智度论》说「
                  佛……
                  入般涅槃,
                  卧北首,
                  大地震动。
                  诸三学人,
                  佥然不乐,
                  郁伊交滋。
                  诸无学人,
                  但念诸法,
                  一切无常。
            」
    }%
    。
」

%% ----------------------------------------------------------------------------
\switchcolumn

% %% Jy

% %% 妖
% 张玄之和顾敷分别是顾和的外孙和内孙,
% 他俩小小年纪就聪颖过人,
% 顾和一起教导他们,
% 常常说顾敷要强于张玄之。
% 因为顾和偏爱亲孙,
% 张玄之颇为不高兴。
% 当张玄之九岁,
% 顾敷七岁时,
% 顾和带着他俩一起去佛寺,
% 看到佛陀涅槃像,
% 众弟子有哭泣的,
% 也有不哭泣的,
% 顾和就问两个孩子其中的原因。
% 张玄之说:“
%  与佛祖亲近的就哭泣,
%  与佛祖疏远的就冷淡。”
% 顾敷却说:“
%  不是这样的。
%  开悟的弟子得以忘情,
%  而挣扎于轮回中的弟子才会哭泣。”
