
\switchcolumn[0]*[\section{}]

顾司空未知名,
诣王丞相%
\footnote{%
    顾司空:
        顾和,字君孝。
        少年成才,
        顾荣曾说他「
            吾家之骐骥也,
            必振衰族
        」。
        王导任扬州刺史时,
        召他为从事,
        累迁尚书令,
        死后追赠司空。
}%
。
丞相小极,
对之疲睡%
\footnote{%
    极:疲乏。
    疲睡:打瞌睡。
}%
。
顾思所以叩会之,
因谓同坐曰:「
    昔
    每闻元公道公协赞中宗,
    保全江表%
    \footnote{%
        元公:
            指顾荣,
            他是顾和的族叔。
            颐荣死后,
            溢号为元,
            所以称为元公。
        中宗:
            晋元帝的庙号。
            按:
            顾和初出仕是在元帝时,
            还不可能有元帝的庙号。
        江表:
            长江之外,即江南。
    }%
    。
    体小不安,
    令人喘息。
」
丞相因觉,
谓顾曰:「
    此子珪璋特达,
    机警有锋%
    \footnote{%
        珪璋(guī zhāng)特达:
            珪和璋是玉器,
            是诸侯朝见天子时所用的重礼。
            用珪璋时可以单独送达,
            不须加上别的礼品为辅。
            后用来比喻
            有才德的人
            不用别人推荐也会有成就。
    }%
    。
」

%% ----------------------------------------------------------------------------
\switchcolumn

% %% Jy

% %% 妖

