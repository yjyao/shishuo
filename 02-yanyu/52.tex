
\switchcolumn[0]*[\section{}]

庾法畅造庾太尉,
握麈尾至佳%
\footnote{%
    庾法畅:
        当作康法畅,和尚名,
        著有《人物论》,
        自称
        「悟锐有神,才辞通辩」。
    麈(zhǔ)尾:
        拂尘。
        一说形状像羽扇、扇柄左右扎上麈尾(驼鹿尾)毛,
        谈话时借助它来指画。
        魏晋清谈之士喜欢用它。
}%
。
公曰:「
    此至佳,
    那得在?
」
法畅曰:「
    廉者不求,
    贪者不与,
    故得在耳。
」

%% ----------------------------------------------------------------------------
\switchcolumn

% %% Jy

% %% 妖
% 康法畅和尚去拜访庾亮,
% 拿着一柄非常好看的拂尘。
% 庾公问他:“
%  这样好的拂尘,
%  你是怎么把它留住的呢?”
% 法畅回答说:“
%  清廉的人不会向我讨要,
%  而我也不会给予贪婪的人,
%  所以它才一直在我这里。”
