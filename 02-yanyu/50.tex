
\switchcolumn[0]*[\section{}]

孙齐由、齐庄二人%
\footnote{%
    孙齐由:
        孙潜,字齐由,
        孙盛的长子。
        豫章太守殷仲堪遇孙潜,
        逼他做咨议参军,
        孙潜拒不受任,
        烦忧过度而亡。
}%
,
小时诣庾公。
公问齐由何字,
答曰:「字齐由。」
公曰:「欲何齐邪?」
曰:「齐许由。」
齐庄何字?
答曰:「字齐庄。」
公曰:「欲何齐?」
曰:「齐庄周%
\footnote{%
    庄周:
        庄子,名周,字子休(一说子沐),
        战国时人,
        与老子同是道家学派的代表人物。
}%
。」
公曰:「
    何不慕仲尼而慕庄周?
」
对曰:「
    圣人生知,
    故难企慕%
    \footnote{%
        企慕:仰慕。
    }%
    。
」
庾公大喜小儿对。

%% ----------------------------------------------------------------------------
\switchcolumn

% %% Jy

% %% 妖
% 孙齐由,孙齐庄两兄弟在他们儿时拜见过庾亮。
% 庾公问孙齐由的字是什么,
% 齐由回答:“我的表字是齐由。”
% 又问:“是打算与谁看齐呢?”
% 齐由答道:“希望能和许由比肩。”
% 庾公开始询问齐庄的表字,
% 齐庄回答:“表字齐庄。”
% 又问:“想与哪位前人一样呢?”
% 齐庄答道:“希望能够与庄周相比。”
% 庾公说:“为什么你仰慕庄子而非孔子呢?”
% 齐庄说:“孔圣人生而知之,所以我难以企及啊。”
% 庾公非常喜欢齐庄的回答。
