
\switchcolumn[0]*[\section{}]

会稽贺生,
体识清远,
言行以礼%
\footnote{%
    贺生:
        贺循,字彦先,
        会稽郡人、
        曾任吴国内史、太子太傅。
        生,对读书人的称呼。
    体识:
        禀性见识。
}%
。
不徒东南之美,
实为海内之秀%
\footnote{%
    不徒:不只。
    按:
    《晋书·顾和传》载,
    这两句是王导称赞顾和的话。
    可能《世说新语》另有所本。
}%
。

%% ----------------------------------------------------------------------------
\switchcolumn

% %% Jy
% 会稽有个书生叫贺循,
% 本性清雅,
% 见识深远,
% 言语得体,
% 彬彬有礼。
% 不单是东南的才子,
% 更算是海内的俊秀。

% %% 妖
% 会稽郡的贺循,
% 禀性清高见识深远,
% 一言一行都很有礼数。
% 他不仅是东南的俊杰,
% 而且在全国范围内都算得上人才。
