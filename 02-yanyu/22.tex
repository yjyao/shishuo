
\switchcolumn[0]*[\section{}]

蔡洪赴洛%
\footnote{%
    蔡洪:
        字叔开,吴郡人,
        原在吴国做官,吴亡后入晋,
        公认才华出众,
        西晋初年太康年间,
        由本州举荐为秀才,
        到京都洛阳。
}%
,
洛中人问曰:「
    幕府初开,
    群公辟命,
    求英奇于仄陋,
    采贤俊于岩穴%
    \footnote{%
        幕府:
            原指将军的官署,也用来指军政大员的官署。
        群公:
            众公卿,指朝廷中的高级官员。
        辟命:
            征召。
        仄陋:指出身贫贱的人。
        岩穴:山中洞穴。
    }%
    。
    君吴、楚之士,
    亡国之余,
    有何异才
    而应斯举%
    \footnote{%
        吴楚:
            春秋时代的吴国和楚国。
            两国都在南方,所以也泛指南方。
    }%
    ?
」
蔡答曰:「
    夜光之珠,不必出于孟津之河;
    盈握之璧,不必采于昆仑之山%
    \footnote{%
        夜光之珠:
            即夜明珠,
            是春秋时代隋国国君的宝珠,又叫隋侯珠,或称隋珠,
            传说是一条大蛇从江中衔来的。
        孟津:
            渡口名,在今河南省盂县南。
            周武王伐纣时和各国诸侯在这里会盟,是一个有名的地方。
        盈握:
            满满一把。这里形容大小。
        壁:中间有孔的圆形玉器。
        昆仑:
            古代盛产美玉的山。
    }%
    。
    大禹生于东夷,
    文王生于西羌%
    \footnote{%
        大禹:
            夏代第一个君主,传说曾治平洪水。
        东夷:
            我国东部的各少数民族。
        文王:
            周文王,
            殷商时一个诸侯国的国君,封地在今陕西一带。
        西羌:
            我国西部的一个民族。
    }%
    。
    圣贤所出,何必常处。
    昔武王伐纣,
    迁顽民于洛邑,
    得无诸君是其苗裔乎%
    \footnote{%
        「昔武王」句:
            周武王灭了殷纣以后,
            把殷朝的顽固人物迁到洛水边上,
            派周公修建洛邑安置他们。
            战国以后,洛邑改为洛阳。
        得无:
            莫非。
    }%
    ?
」

%% ----------------------------------------------------------------------------
\switchcolumn

% %% Jy

% %% 妖

