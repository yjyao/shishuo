
\switchcolumn[0]*[\section{}]

竺法深在简文坐%
\footnote{%
    竺法深:
        和尚名。
    简文:
        晋简文帝司马昱,
        据记载简文帝当时还没有登帝位,
        只是封为会稽王。
}%
,
刘尹问:「
    道人何以游朱门%
    \footnote{%
        刘尹:
            刘惔。
        朱门:
            红漆的大门,指达官贵人之家。
    }%
    ?
」
答曰:「
    君自见朱门,
    贫道如游蓬户%
    \footnote{%
        蓬户:
            用蓬草编成的门,指简陋的房屋,穷苦人家。
    }%
    。
」
或云卞令%
\footnote{%
    卞令:
        卞壶,字望之,曾任尚书令。
}%
。

%% ----------------------------------------------------------------------------
\switchcolumn

% %% Jy
% 竺法深应召去诣见简文帝,
% 丹阳尹刘真长问道:「
%     大师怎么在达贵家中往来?
% 」
% 竺法深说:「
%     先生看到的是豪宅丽室,
%     贫僧却只当是蓬门穷户。
% 」
% 也有人说发问者其实是尚书令卞壶。

% %% 妖
% 和尚竺法深成为了简文帝的座上宾,
% 刘惔问他:“
%  大师您怎么和官宦人家交往呢?”
% 和尚回答说:“
%  您自己看到那是达官显贵,
%  但对我来说这与蓬门小户相交也没有差别。”
% 有人说,这可能是卞壶问的,而非刘惔。
