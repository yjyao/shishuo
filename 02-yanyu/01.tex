
\switchcolumn[0]*[\section{}]

边文礼见袁奉高%
\footnote{%
    边文礼:边让,字文礼,才俊辩逸。
            后任九江太守,被魏武帝曹操杀害。
}%
,
失次序%
\footnote{%
    失次序:失顺序,不合礼节。即举止失措,举动失常。
}%
。
奉高曰:「
    昔尧聘许由,
    面无怍色%
    \footnote{%
        尧聘许由,面无怍色:
            尧是传说中的远古帝王,
            许由是传说中的隐士。
            尧想让位给许由,
            许由不肯接受。
            尧又想请他出任九州长,
            许由认为
            这些名禄之言
            污了他的耳朵,
            就跑去洗耳。
        怍(zuò)色:羞愧的脸色。
    }%
    。
    先生何为颠倒衣裳%
    \footnote{%
        颠倒衣裳:把衣和裳掉过来穿,后用来比喻举动失常。
                  衣,上衣;裳,下衣,是裙的一种,古代男女都穿裳。
                  这句话出自《诗经·齐风·东方未明》:
                  「东方未明,颠倒衣裳。」
    }%
    ?
」
文礼答曰:「
    明府初临,
    尧德未彰,
    是以
    贱民颠倒衣裳耳%
    \footnote{%
        明府:高明的府君,吏民也称太守为明府。
        尧德:如尧之德;大德。
              袁奉高说到「尧聘许由」之事,
              所以边文礼也借谈「尧德」来嘲讽他。
        贱民:袁奉高似乎曾任陈留郡太守,
              而边文礼是陈留人,
              所以谦称为贱民。
    }%
    。
」

%% ----------------------------------------------------------------------------
\switchcolumn

边文礼谒见袁奉高时,
手忙脚乱,有失礼节。
袁奉高说:「
    古时候
    帝尧禅位给许由,
    可没见许由
    惊慌失措、面露愧色啊。
    先生怎么
    弄得这么忙乱呢?
」
文礼回答道:「
    大人来陈留到任
    不过几天,
    这不还没来得及彰显出
    像帝尧时那样的大德来嘛。
    所以小的
    也就难免紧张失态了。
」
