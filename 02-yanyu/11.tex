
\switchcolumn[0]*[\section{}]

钟毓、钟会少有令誉%
\footnote{%
    钟毓(yù)、钟会:
        是兄弟俩。
        钟毓,字稚叔,
        少年成才,
        十四岁任散骑侍郎,
        后升至车骑将军。
        钟会,字士季,
        自幼聪慧,
        被看成是非常人物。
        他博学多闻,足智多谋,
        后迁镇西将军,
        平定蜀汉,
        又进位司徒。
        他自认为
        「功名盖世,不可复为人下」,
        因而谋划反帝室,
        却因兵变而败,
        死于乱军。
    按:
        该则故事真实性有疑点。
        程炎震提出,根据《毓传》,
        钟毓十四岁就开始为官,
        而有太和年初期(227 年左右)向魏明帝上奏的记载,
        所以魏文帝时(220 -- 226 年)
        钟毓差不多已经十四岁了。
        而根据钟会自己的阐述,
        他黄初六年(225 年)才出生,
        文帝 226 年就已经离世,
        更不可能见过钟会。
        等到钟会十三岁时,
        其父钟繇也已经过世,
        也不可能为他引见面圣。
}%
,
年十三,
魏文帝闻之,
语其父钟繇曰%
\footnote{%
    钟繇(yáo):
        字元常,
        文帝时任太尉,
        与司徒华歆、司空王朗并称三公。
}%
:
「可令二子来。」
于是敕见%
\footnote{%
    敕(chì):皇帝的命令。
}%
。
毓面有汗,
帝曰:
「卿面何以汗?」
毓对曰:
「战战惶惶,汗出如浆。」
复问会:
「卿何以不汗?」
对曰:
「战战栗栗,汗不敢出 。」

%% ----------------------------------------------------------------------------
\switchcolumn

% %% Jy
% 钟毓、钟会兄弟俩
% 从小就享有美名。
% 十三岁的时候,
% 魏文帝听说了他们,
% 就让他们的父亲钟繇
% 带他们来面圣。
% 兄弟二人来在文帝面前,
% 文帝见钟毓
% 额上泛着汗珠,
% 便问「
%     爱卿怎么出汗了?
% 」
% 钟毓答道:「
%     下臣诚恐诚惶,
%     汗发起来像喷浆。
% 」
% 文帝于是又问钟会:「
%     那爱卿你
%     怎么不出汗呢?
% 」
% 钟会答道:「
%     下臣栗栗恓恓,
%     汗也流不出一滴。
% 」

% %% 妖
% 钟毓、钟会两个人在少年的时候就颇有美名,
% 他们十三岁时,
% 魏文帝就听说了他俩,
% 对他们的父亲钟繇说:“
%  让你两个儿子来觐见吧。”
% 面圣时钟毓脸上有汗,
% 魏文帝问:“
%  你脸上为何出汗呢?”
% 钟毓回答:“
%  战战惶惶,汗出如浆。”
% 过了一会又问:“
%  你为何又不出汗了呢?”
% 钟毓答道:“
%  战战栗栗,汗不敢出。”
