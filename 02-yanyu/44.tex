
\switchcolumn[0]*[\section{}]

孔廷尉以裘与从弟沈%
\footnote{%
    孔廷尉:指孔君平。
    从弟:堂弟。
    沈:
        孔沈,字德度,
        受举荐,被任琅邪王文学,
        拒不受。
}%
,
沈辞不受。
廷尉曰:「
    晏平仲之俭,
    祠其先人,
    豚肩不掩豆,
    犹狐裘数十年,
    卿复何辞此%
    \footnote{%
        晏平仲:
            晏婴,谥平,字仲,
            春秋时代齐国大夫,
            主张节俭,
            据说他一件狐裘穿了三十年。
        豚:小猪。
        豆:盛食物的器具,形似高脚盘。
    }%
    !
」
于是受而服之。

%% ----------------------------------------------------------------------------
\switchcolumn

% %% Jy
% 廷尉孔君平
% 送了一件裘大衣
% 给堂弟孔沈,
% 孔沈推辞不要。
% 廷尉说:「
%     你想想
%     春秋时的晏婴
%     多么地节俭,
%     他祭祀祖先,
%     用的猪仔都放不满一个豆盘,
%     就这么地,
%     自己还有一件穿了几十年的孤裘呢。
%     你又何必连一件衣服都不肯收下?
% 」
% 孔沈这才把大衣收了下来。

% %% 妖
% 廷尉孔君平送了一件皮裘大衣给他的堂弟孔沈,
% 孔沈推辞了他的好意,拒不接受。
% 孔君平说:“
%  齐国大夫晏婴那样节俭的人,
%  去祠堂祭司祖先时,
%  猪肘撑开都盖不满供盘,
%  还穿了几十年狐狸毛大衣,
%  你又为什么不肯收下这件呢!”
% 孔沈这才同意留下皮衣。
