
\switchcolumn[0]*[\section{}]

高坐道人
不作汉语%
\footnote{%
    高坐:
        西域和尚名,
        西晋永嘉年间到中国。
        据说是
        云国的王子,
        把王位让给了弟弟,
        自己出了家。
        据《高坐别传》载:他「
            性高简,
            不学晋语。
            诸公与之言,
            皆因传译。
        」
        但是能领会神色,
        往往还没有说话,
        就已经领悟了对方的意图。
    道人:
        指和尚。
}%
。
或问此意,
简文曰%
\footnote{%
    简文:
        晋简文帝司马昱。
}%
:「
    以简应对之烦。
」

%% ----------------------------------------------------------------------------
\switchcolumn

% %% Jy

% %% 妖
% 高坐和尚不说汉语。
% 有人问其缘由,
% 晋简文帝说:“
%  这是为了省去应酬的烦恼。”
