
\switchcolumn[0]*[\section{}]

孔融被收,
中外惶怖%
\footnote{%
    孔融被收:指孔融被曹操逮捕一事。
              孔融为人恃才负气,多次反对曹操的决定,
              曹操借孔融反礼教、反孝道的文章为由
              将孔融满门抄斩。
    中外:指朝廷内外。
}%
。
时融儿大者九岁,小者八岁,
二儿
故琢钉戏%
\footnote{%
    琢钉戏:一种小孩玩的游戏。
            清朝周亮工《因树屋书影》载「
                金陵童子有琢钉戏,
                画地为界,
                琢钉其中,
                先以小钉琢地,
                名曰签,
                以签之所在为主。
                出界者负,
                彼此不中者负,
                中而触所主签亦负。
            」
}%
,
了无遽容%
\footnote{%
    遽(jù):惊惧、慌张。
}%
。
融谓使者曰:「
    冀罪止于身%
    \footnote{%
        冀:希望。
    }%
    ,
    二儿可得全不?
」
儿徐进曰:「
    大人岂见覆巢之下,
    复有完卵乎?
」
寻亦收至%
\footnote{%
    寻:不久。
    收:指前来收押的差役。
}%
。

%% ----------------------------------------------------------------------------
\switchcolumn

% %% Jy
% 孔文举被曹操抓走以后,
% 朝廷内外
% 惶恐不安。
% 那时孔文举的两个儿子
% 一个九岁、一个八岁。
% 见到父亲被捕,
% 两个小孩儿
% 面无异色,
% 还照样不慌不忙地
% 玩儿着他们的琢钉戏。
% 孔文举对差役说:「
%     罪都在我的身上,
%     能不能请你们放过我的孩子们?
% 」
% 他的儿子从容地说:「
%     父亲,
%     鸟巢打翻了,
%     底下的蛋还能不破吗?
% 」
% 两个孩子很快也被抓去了。

% %% 妖

% 孔融被曹操逮捕之后,
% 朝廷内外都很惶恐害怕。
% 这时他大儿子九岁,小儿子八岁,
% 两个孩子依然在玩琢钉戏,
% 丝毫没有慌张的样子。
% 孔融对使者说:“
%  希望只惩罚我,
%  我的两个孩子能保全吗?”
% 他的儿子缓缓走上前说:“
%  您见过鸟巢打翻之后还有完整的鸟蛋嘛?”
% 不就之后,来收押孔融儿子的差役就也到了。
