
\pagebreak
\switchcolumn[0]*[\section{}]

挚瞻曾作
四郡太守、
大将军户曹参军,
复出作内史%
\footnote{%
    挚瞻:
        字景游,
        西晋末,
        在王敦的大将军幕府中任户曹参军,
        历任多郡太守。
        后与王敦言语不合,
        被贬为随国内史
        (王侯封国中的官职,与太守相当)。
}%
。
年始二十九。
尝别王敦,
敦谓瞻曰:「
    卿年未三十,
    已为万石,
    亦太蚤%
    \footnote{%
        万石(dàn):
            石,容量单位,为十斗。
            表示官职等级是由俸谷多少来定的,
            太守是二千石。
            挚瞻曾作四郡太守,现又作内史,共五郡,
            所以说万石。
        蚤:
            即「早」。
    }%
    。
」
瞻曰:「
    方于将军,
    少为太蚤;
    比之甘罗,
    已为太老%
    \footnote{%
        方:
            相比。
        少:
            稍;略微。
        甘罗:
            战国时秦人,
            十二岁为秦外交使节,
            封为上卿。
    }%
    。
」

%% ----------------------------------------------------------------------------
\switchcolumn

% %% Jy
% 挚瞻曾当过
%     四郡的太守、
%     大将军的户曹参军,
% 现如今又要出去做内史,
% 但是年纪轻轻,
% 方才二十九。
% 他去向大将军王敦告别,
% 王敦说他还不满三十岁
% 就已经做了俸值万石的官了,
% 「
%     也是太早。
% 」
% 王敦说。
% 挚瞻说:「
%     和将军相较,确实是稍早了一些;
%     但若是
%     和甘罗相比,那我已经太老太老了。
% 」
% % 挚瞻卒。

% %% 妖
% 挚瞻先后担任过四郡的太守、
% 王敦军中的户曹参军,
% 后来又被贬为内史。
% 这个时候他才二十九岁。
% 他曾经向王敦告别,
% 王敦对挚瞻说:“
%  您尚未而立,
%  就已做过五任薪水高达两千石的官职,
%  真的太年轻了。”
% 挚瞻回答说:“
%  和将军您比呢,
%  我确实是稍微早了一些;
%  但若是和那十二岁做宰相的甘罗相比,
%  我实在是马齿徒增啊。”
